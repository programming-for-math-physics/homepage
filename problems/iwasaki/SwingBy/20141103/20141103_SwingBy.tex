
%\documentclass{jsarticle}

\documentclass{article}
\usepackage{amsmath}
\usepackage[dvipdfmx]{graphicx,color}

%一段落目でインデントしたい
\usepackage{indentfirst}

\usepackage{comment}
\usepackage{ascmac}
\usepackage{otf}
\usepackage{url}
\usepackage{setspace}
\usepackage{wrapfig}

\usepackage[top=20truemm, bottom=20truemm, left=20truemm, right=20truemm]{geometry}

\usepackage{here}
%\begin{figure}[H]とかやると、強制的に位置を文章の一定位置に固定する。

\usepackage{caption}

\usepackage{listings,multicol}
\lstset{
	language={C++}, 
	basicstyle={\ttfamily\footnotesize},
		commentstyle={\small\ttfamily \color[rgb]{0,0.5,0}},
		keywordstyle={\small\bfseries \color[rgb]{0,0,1}},
		ndkeywordstyle={\small\ttfamily  \color[rgb]{0.5,0.5,0.0}}, %特殊キーワード、かな?
	alsoletter={\#},%無理やり#defineをキーワード化するため。
	morekeywords={\#define},%普通のキーワード
	morendkeywords={spawn,sync},%別の色で着色できる、特別キーワード
	showstringspaces=false
	identifierstyle={\small},
	stringstyle={\small\ttfamily}, 
	frame={tb},%singleとかで外側を線で囲む。tbで上下だけ、noneでなし
	breaklines=true,%折り返し
	columns=[l]{fixed},
	numbers=none,%これは、行番号。leftとかで左に行番号がつく
	xrightmargin=0zw,
	xleftmargin=3zw,
	numberstyle={\scriptsize},
	stepnumber=1,%行番号増分
	numbersep=1zw,
	lineskip=-0.25ex,
	morecomment=[l]{//},
	escapeinside={<@}{@>},
	tabsize=2,
}

\usepackage{subfig}

\captionsetup[subfloat]{position=bottom,
  farskip=10pt,topadjust=0pt,captionskip=10pt,
  nearskip=10pt,margin=10pt}

\renewcommand{\figurename}{Figure}
\renewcommand{\tablename}{}

%ページ全体の行間
\setstretch{1.0}
%ページ全体の段落間
\parskip=3pt plus 1pt

\title{ スイングバイ 参考資料 v1.0 }
\author{ 岩崎 慎太郎 }

\begin{document}

\date{November 3, 2014}
\maketitle

%%%%%%%%%%%%%%%%%%%%%%%%%%%%%%%%%%%%%%%%%%%%%%%%%%%%%%%%%%%%%%%%%%%%%%%%%%%%%%%%%%%%%%%%%%%%%%%%%%%%%%%%%%%%
\section{ 問題 }
%%%%%%%%%%%%%%%%%%%%%%%%%%%%%%%%%%%%%%%%%%%%%%%%%%%%%%%%%%%%%%%%%%%%%%%%%%%%%%%%%%%%%%%%%%%%%%%%%%%%%%%%%%%%

太陽系外に探査機を打ち出すことを考える。
打ち上げ日時(と必要があれば位置)を求めよ。
ただし、1975年1月1日から1980年になるまで(ボイジャーが打ち上げられた年代)に打ち上げるとする。

シミュレーションの条件は以下のように置くが、合理的な範囲で適宜条件の追加および削除をしてよい。
また、計算に必要な定数は各自で補うこと。

\begin{itemize}
\item 太陽の重力場を脱出できる(無限遠まで行ける)運動エネルギーを探査機が得られればよい
\item 地上からの発射時の初速を$14km$で、ロケットからの分離などは考えない
\item 探査機の重さは$750kg$とする
\item 惑星は太陽を中心に円軌道を描き、すべて同一平面状で運動しているとする
\item 探査機は、地球を黄道面で切ったときの断面円周上のどこから打ち上げてもよい
\end{itemize}

特に必要がなければ、地球の自転の影響を無視して地上から垂直に打ち上げたと考えてよい。

%%%%%%%%%%%%%%%%%%%%%%%%%%%%%%%%%%%%%%%%%%%%%%%%%%%%%%%%%%%%%%%%%%%%%%%%%%%%%%%%%%%%%%%%%%%%%%%%%%%%%%%%%%%%
\section{ 小問題 }
%%%%%%%%%%%%%%%%%%%%%%%%%%%%%%%%%%%%%%%%%%%%%%%%%%%%%%%%%%%%%%%%%%%%%%%%%%%%%%%%%%%%%%%%%%%%%%%%%%%%%%%%%%%%

%%%%%%%%%%%%%%%%%%%%%%%%%%%%%%%%%%%%%%%%%%%%%%%%%%%%%%%%%%%%%%%%%%%%%%%%%%%%%%%%%%%%%%%%%%%%%%%%%%%%%%%%%%%%
\subsection{ 小問題1 }
%%%%%%%%%%%%%%%%%%%%%%%%%%%%%%%%%%%%%%%%%%%%%%%%%%%%%%%%%%%%%%%%%%%%%%%%%%%%%%%%%%%%%%%%%%%%%%%%%%%%%%%%%%%%

地球と探査機のみを考える。探査機を適切な速度で入射させることで、探査機が楕円軌道を描き、
その焦点が地球になることを示しなさい。
この時、重力ポテンシャルと運動エネルギーの和が保存されること、及び探査機と地球の間にケプラーの第二法則が成り立っていることを確認しなさい。

%%%%%%%%%%%%%%%%%%%%%%%%%%%%%%%%%%%%%%%%%%%%%%%%%%%%%%%%%%%%%%%%%%%%%%%%%%%%%%%%%%%%%%%%%%%%%%%%%%%%%%%%%%%%
\subsection{ 小問題2 }
%%%%%%%%%%%%%%%%%%%%%%%%%%%%%%%%%%%%%%%%%%%%%%%%%%%%%%%%%%%%%%%%%%%%%%%%%%%%%%%%%%%%%%%%%%%%%%%%%%%%%%%%%%%%

地球と探査機のみを考える。第一宇宙速度および第二宇宙速度が、シミュレーションで実現されるか確認しなさい。

%%%%%%%%%%%%%%%%%%%%%%%%%%%%%%%%%%%%%%%%%%%%%%%%%%%%%%%%%%%%%%%%%%%%%%%%%%%%%%%%%%%%%%%%%%%%%%%%%%%%%%%%%%%%
\subsection{ 小問題3 }
%%%%%%%%%%%%%%%%%%%%%%%%%%%%%%%%%%%%%%%%%%%%%%%%%%%%%%%%%%%%%%%%%%%%%%%%%%%%%%%%%%%%%%%%%%%%%%%%%%%%%%%%%%%%

地球と探査機のみを考える。公転速度で直線運動する地球に対して、探査機が適切な速度および軌道で加速スイングバイおよび減速
スイングバイすることを示しなさい。

%%%%%%%%%%%%%%%%%%%%%%%%%%%%%%%%%%%%%%%%%%%%%%%%%%%%%%%%%%%%%%%%%%%%%%%%%%%%%%%%%%%%%%%%%%%%%%%%%%%%%%%%%%%%
\subsection{ 小問題4 }
%%%%%%%%%%%%%%%%%%%%%%%%%%%%%%%%%%%%%%%%%%%%%%%%%%%%%%%%%%%%%%%%%%%%%%%%%%%%%%%%%%%%%%%%%%%%%%%%%%%%%%%%%%%%

小問題3について、地球が常に静止しているような座標系を考えて、解析的に解いた場合と比較しなさい。

%%%%%%%%%%%%%%%%%%%%%%%%%%%%%%%%%%%%%%%%%%%%%%%%%%%%%%%%%%%%%%%%%%%%%%%%%%%%%%%%%%%%%%%%%%%%%%%%%%%%%%%%%%%%
\subsection{ 小問題5 }
%%%%%%%%%%%%%%%%%%%%%%%%%%%%%%%%%%%%%%%%%%%%%%%%%%%%%%%%%%%%%%%%%%%%%%%%%%%%%%%%%%%%%%%%%%%%%%%%%%%%%%%%%%%%

当時の太陽系の惑星の位置・軌道を得て、プログラムで読み込めるようにしなさい。

%%%%%%%%%%%%%%%%%%%%%%%%%%%%%%%%%%%%%%%%%%%%%%%%%%%%%%%%%%%%%%%%%%%%%%%%%%%%%%%%%%%%%%%%%%%%%%%%%%%%%%%%%%%%
\subsection{ 小問題6 }
%%%%%%%%%%%%%%%%%%%%%%%%%%%%%%%%%%%%%%%%%%%%%%%%%%%%%%%%%%%%%%%%%%%%%%%%%%%%%%%%%%%%%%%%%%%%%%%%%%%%%%%%%%%%

太陽系を脱出できるような打ち上げ日時(と位置)を求めなさい。

%%%%%%%%%%%%%%%%%%%%%%%%%%%%%%%%%%%%%%%%%%%%%%%%%%%%%%%%%%%%%%%%%%%%%%%%%%%%%%%%%%%%%%%%%%%%%%%%%%%%%%%%%%%%
\section{ 応用問題例 }
%%%%%%%%%%%%%%%%%%%%%%%%%%%%%%%%%%%%%%%%%%%%%%%%%%%%%%%%%%%%%%%%%%%%%%%%%%%%%%%%%%%%%%%%%%%%%%%%%%%%%%%%%%%%

%%%%%%%%%%%%%%%%%%%%%%%%%%%%%%%%%%%%%%%%%%%%%%%%%%%%%%%%%%%%%%%%%%%%%%%%%%%%%%%%%%%%%%%%%%%%%%%%%%%%%%%%%%%%
\subsection{ 応用問題例1 }
%%%%%%%%%%%%%%%%%%%%%%%%%%%%%%%%%%%%%%%%%%%%%%%%%%%%%%%%%%%%%%%%%%%%%%%%%%%%%%%%%%%%%%%%%%%%%%%%%%%%%%%%%%%%

脱出できるエネルギーを得た上で、最も早い時期に太陽から$50au$離れることができるような発射日時を求めなさい。

%%%%%%%%%%%%%%%%%%%%%%%%%%%%%%%%%%%%%%%%%%%%%%%%%%%%%%%%%%%%%%%%%%%%%%%%%%%%%%%%%%%%%%%%%%%%%%%%%%%%%%%%%%%%
\subsection{ 応用問題例2(困難) }
%%%%%%%%%%%%%%%%%%%%%%%%%%%%%%%%%%%%%%%%%%%%%%%%%%%%%%%%%%%%%%%%%%%%%%%%%%%%%%%%%%%%%%%%%%%%%%%%%%%%%%%%%%%%

1970年代に太陽系外に脱出できる最も小さい初速を求めなさい。

%%%%%%%%%%%%%%%%%%%%%%%%%%%%%%%%%%%%%%%%%%%%%%%%%%%%%%%%%%%%%%%%%%%%%%%%%%%%%%%%%%%%%%%%%%%%%%%%%%%%%%%%%%%%
\section{ もっと詳しく }
%%%%%%%%%%%%%%%%%%%%%%%%%%%%%%%%%%%%%%%%%%%%%%%%%%%%%%%%%%%%%%%%%%%%%%%%%%%%%%%%%%%%%%%%%%%%%%%%%%%%%%%%%%%%

スイングバイは二体問題の場合は解析解が得られるので、理論式と結果とをよく比較することが大切である。
また、計算すべき範囲が広いので、必要であれば適切に計算範囲を制限することも重要である。

より精度の高いルンゲ=クッタ法を用いて、シミュレーションの時間幅$\Delta$を大きくするなど、
計算時間の短縮も検討する必要がある。

今回は簡単のため省略したが、実際は探査機自体もエンジンを積み、多少の方向調整ができるものもある。

%%%%%%%%%%%%%%%%%%%%%%%%%%%%%%%%%%%%%%%%%%%%%%%%%%%%%%%%%%%%%%%%%%%%%%%%%%%%%%%%%%%%%%%%%%%%%%%%%%%%%%%%%%%%
\section{ 参考となるWebページ }
%%%%%%%%%%%%%%%%%%%%%%%%%%%%%%%%%%%%%%%%%%%%%%%%%%%%%%%%%%%%%%%%%%%%%%%%%%%%%%%%%%%%%%%%%%%%%%%%%%%%%%%%%%%%

\begin{itemize}
\item[1.] NASA「Heliocentric Trajectories for Selected Planets」\\ \url{http://omniweb.gsfc.nasa.gov/coho/helios/planet.html}
       \\ ある時点での太陽系の惑星の位置などが分かります。
\end{itemize}


\end{document}
