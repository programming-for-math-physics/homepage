
\documentclass[unicode,12pt]{beamer}
\usepackage[no-math]{luatexja-fontspec}
\usepackage{graphicx}
\usepackage{tcolorbox}
\usepackage{xifthen}
\usepackage{tikz}
\usepackage{listings}
\usepackage{bm}

% Looks
\setbeamertemplate{navigation symbols}{}
\renewcommand{\baselinestretch}{1.25}
\setbeamersize{text margin left=15pt,text margin right=15pt}

% Fonts
\setsansfont[
    BoldFont=OpenSans-Bold,
    FontFace={sb}{n}{Font=OpenSans-Semibold},
]{OpenSans-Regular}
\setmainjfont[
    BoldFont=Gen Shin Gothic P Bold,
    YokoFeatures={JFM=prop},
    FontFace={sb}{n}{Font=Gen Shin Gothic P Medium},
]{Gen Shin Gothic P Normal}
\setmonofont{inconsolata}
\usepackage{newtxmath}
\usefonttheme{professionalfonts}
\setbeamerfont{structure}{series=\fontseries{sb}}
\setbeamerfont{alerted text}{series=\bfseries}
\setbeamerfont{frametitle}{size=\large,series=\bfseries}
\setbeamerfont{title}{size=\Large,series=\bfseries}
\setbeamerfont{subtitle}{size=\normalsize,series=\fontseries{sb}}
\setbeamerfont{author}{size=\normalsize,series=\fontseries{sb}}
\setbeamerfont{institute}{size=\small}

% Color
\definecolor{WhiteBlue}{HTML}{f9fdff}
\definecolor{LightBlue}{HTML}{dbf2ff}
\definecolor{Blue}{HTML}{2d5ae0}
\definecolor{DarkBlue}{HTML}{143447}

\definecolor{WhiteOrange}{HTML}{fffcf9}
\definecolor{LightOrange}{HTML}{ffeee0}
\definecolor{Orange}{HTML}{e33900}
\definecolor{DarkOrange}{HTML}{4f2604}

\definecolor{Purple}{HTML}{9400d3}
\definecolor{WhitePurple}{HTML}{faf0ff}
\definecolor{DarkPurple}{HTML}{441c58}

\definecolor{Green}{HTML}{009e73}
\definecolor{WhiteGreen}{HTML}{eefffa}
\definecolor{DarkGreen}{HTML}{144437}

\setbeamercolor{normal text}{fg=DarkGreen}
\setbeamercolor{structure}{fg=Green}
\setbeamercolor{alerted text}{fg=Purple}
\setbeamercolor{example text}{fg=Green}
\setbeamercolor{boxdefault}{fg=WhiteGreen,bg=Green}
\setbeamercolor{boxalerted}{fg=WhiteOrange,bg=Purple}
\setbeamercolor{section in toc shaded}{fg=DarkGreen}

% Footer
\setbeamertemplate{footline}[frame number]
\setbeamerfont{footline}{size=\normalsize}

% Framed Block
\defbeamertemplateparent{blocks}[framed]{block begin,block end}[1][]
{[#1]}
\defbeamertemplate{block begin}{framed}[1][]{
    \begin{tcolorbox}[
        colback=WhiteGreen,
        colframe=Green,
        colbacktitle=Green,
        coltitle=WhiteGreen,
        coltext=DarkGreen,
        sharp corners,
        boxrule=0.7pt,
        title=\fontseries{sb}\selectfont\insertblocktitle
    ]
    \usebeamerfont{block body}
}
\defbeamertemplate{block end}{framed}[1][]{\end{tcolorbox}}
\setbeamertemplate{blocks}[framed]
\defbeamertemplateparent{blocks alerted}[framed]{block alerted begin,block alerted end}[1][]
{[#1]}
\defbeamertemplate{block alerted begin}{framed}[1][]{
    \begin{tcolorbox}[
        colback=WhitePurple,
        colframe=Purple,
        colbacktitle=Purple,
        coltitle=WhitePurple,
        coltext=DarkPurple,
        sharp corners,
        boxrule=0.7pt,
        title=\bfseries\insertblocktitle
    ]
    \usebeamerfont{block alerted body}
}
\defbeamertemplate{block alerted end}{framed}[1][]{\end{tcolorbox}}
\setbeamertemplate{blocks alerted}[framed]

% Title Frame
\setbeamertemplate{frametitle}{
    \vspace{-1mm}
    \usebeamerfont{frametitle}\usebeamercolor[fg]{frametitle}\insertframetitle\par
    \vspace{-4.5mm}\hspace{-3mm}
    \begin{tikzpicture}
        \draw (0,0) -- (10,0);
    \end{tikzpicture}
}

% Title
\setbeamertemplate{title page}{
    \begin{flushright}
        {
            \usebeamerfont{title}\usebeamercolor[fg]{title}\inserttitle
            \vspace{-4mm}
            \begin{tikzpicture}
                \draw (0,0) -- (10,0);
            \end{tikzpicture}
        } \\
        \vspace{-4mm}
        {
            \usebeamerfont{subtitle}\usebeamercolor[fg]{subtitle}
            \ifthenelse{\isempty{\subtitle}}{}{\vspace{2mm}\insertsubtitle}
        } \\
        \vspace{8mm}
        {
            \usebeamerfont{author}\usebeamercolor[fg]{author}
            \ifthenelse{\isempty{\author}}{}{\insertauthor}
        } \\
        {
            \usebeamerfont{institute}\usebeamercolor[fg]{institute}
            \ifthenelse{\isempty{\institute}}{}{\vspace{1mm}\insertinstitute}
        } \\
        \vspace{8mm}
        {
            \usebeamerfont{date}\usebeamercolor[fg]{date}
            \ifthenelse{\isempty{\date}}{}{\insertdate}
        }
    \end{flushright}
}

% Listings
\lstset{
    language=C++,
    basicstyle=\ttfamily\footnotesize,
    keywordstyle=\bfseries\color{Green},
    stringstyle=\color{DarkOrange},
    commentstyle=\color{Purple},
    showstringspaces=false,
    breaklines=true,
    columns=[l]{fullflexible},
    lineskip=2pt,
}

% Itemize
\let\OLDitemize\itemize
\renewcommand{\itemize}{\OLDitemize\setlength{\itemindent}{-5pt}}
\let\OLDdescription\description
\renewcommand{\description}{\OLDdescription\setlength{\itemindent}{-20pt}}
\setbeamertemplate{itemize items}[circle]

% Biblatex
\usepackage[backend=bibtex,style=ieee]{biblatex}
\addbibresource{reference.bib}
\AtEveryCitekey{\iffootnote{\scriptsize}{}}
\setbeamertemplate{bibliography item}[text]

% Caption
\setbeamerfont{caption}{size=\footnotesize}
\setbeamertemplate{caption}[numbered]
\setbeamertemplate{caption label separator}{}
\setlength\abovecaptionskip{-5pt}
\renewcommand{\figurename}{Fig.}

% misc
\newtheorem{thm}{定理}
\newtheorem{lem}{補題}


\title{ノイズキャンセリング}
\author{TA~~~遠藤 亘}
\date{2014-11-18}

\begin{document}

\maketitle

\section{問題}

ある直方体の空間中に騒音源があり,
これによって発生する騒音を空間内部のある領域
において除去したい.
騒音源は周波数,位相,振幅が全て既知の正弦波を発生させるとし,
消音スピーカーを適宜配置することで,
特定の領域に発生する音波のエネルギーを最小化せよ.

\section{小問題}

\subsection{基本方程式の導出}

波動は,流体以外にも電磁波のような異なる物理現象においても,
波動方程式と呼ばれる一意な式によって表せる.

空気の``運動方程式''と,``連続の式''から,適切な線形近似を行い,
波動方程式を導出せよ.

\subsection{FDTD法の適用}

偏微分方程式を数値的に解く方法として,
基本的な手法として差分法が知られている.
特に,時間微分と空間微分の両方を差分近似する手法はFDTD法と呼ばれ,
波動方程式の最も基本的な数値解析手法である.

上で求めた波動方程式から,
FDTD法による数値計算に必要な時間ステップ毎の更新式を導出せよ.

数値シミュレーションをする上では,
更新式だけでなく,初期条件と境界条件も重要な要素である.
特に音波が壁で反射するような状況を考慮すると,
境界条件はどのように指定すればいいだろうか.

\subsection{FDTD法のプログラム}

FDTD法によって実際に数値計算を行うプログラムを記述し,
音波伝搬の様子をグラフ化せよ.

まずは,最も単純に,ある点の周りの6点の値を基に
差を計算するプログラムを記述して,
きちんと波が伝搬することを確認するとよい.
しかし,これでは実行性能に問題があるので,
改善手法について検討せよ.

\subsection{エネルギー最小化}

FDTD法による音波伝搬シミュレータを用いて,
スピーカーのパラメータを複数試し,
音を最小化出来るパラメータを
求めるプログラムを作成せよ.

\section{発展課題}

\subsection{格子モデルの改良}

同じFDTD法によるシミュレーションであっても,
格子点をずらしたスタッガード格子によって、
シミュレーションの誤差を低減できることがあること
が知られている.

先ほどまでの通常格子に加えて,
スタッガード格子を利用したプログラムも作成し,
両者の結果を比較せよ.

\subsection{PyAudioを試す}

Pythonには,波形を実際に音声として再生するPyAudioというモジュールがある.
これまでシミュレーションした結果について,
空間のどこか1点について実際の音として再生してみよ.

余裕があれば,ドップラー効果のような
波動現象について,実際にシミュレーションで再現できているかどうか確認できるとよい.


\subsection{任意波形への拡張 (やや難しい)}

これまでは騒音を正弦波であると仮定してきたが,
任意波形に拡張することを考えてみよ.
簡単のため,消音すべき場所は1点のみでよく,
消音スピーカーも1つだけとする.


\subsection{適応的なノイズキャンセリング (難しい)}

マイクロフォンで集音して,その波形を基に
スピーカーから音を出力すれば,
入力波形が未知であったとしても
ノイズキャンセリングを行えると期待される.
未知波形のノイズキャンセリングについて検討せよ.


\subsection{非線形音響現象 (おそらくとても難しい)}

空気は実際には完全に線形な振る舞いをするわけではなく,
僅かではあるが非線形な成分が含まれている.
このような非線形現象を応用した例として,
パラメトリックアレイスピーカー等が知られている.

非線形な音波伝搬を解析するには,
通常は線形近似してしまう式をより厳密に計算する必要がある.
非線形現象の解析には,
音響分野にも最近応用され始めたCIP法等,
いくつかの数値計算手法が存在する(らしい).


\section{解説}


\subsection{波動方程式の導出}

\subsubsection{体積と圧力の関係}

ある体積要素について,微小時間における体積$V$と圧力$P$の変化の関係
について見ていく.

まず,微小時間の変化が``断熱過程''であるという仮定をおく.
すなわち,微小時間における体積要素内外の熱のやり取りは無いと仮定する.
この時,
\begin{align}
PV^{\gamma}={\rm const.}
\end{align}
が成り立つ.ただし$\gamma$は比熱比である.
ここで,圧力が$P_0\rightarrow P_0+\Delta p$,体積が$V_0\rightarrow V_0+\Delta v$と変化するなら,
\begin{align}
P_0V_0^\gamma=(P_0+\Delta p)(V_0+\Delta v)^\gamma
\end{align}
である.2次の微小項を取り除くことで,
\begin{align}
\frac{\Delta p}{P_0}=-\gamma \frac{\Delta v}{V_0}
\end{align}
が導かれる.これは,力 (圧力)と 幾何的距離 (体積)の比例関係を示しており,
フックの法則の一種である.
つまり,気体が弾性体として振る舞うことを示している.

\subsubsection{連続の式}

流体における質量保存の法則は,連続の式と呼ばれる.

ここから,ある点$(x, y, z)$の速度を${\bm u}(x, y, z)=(u_1, u_2, u_3)$,
密度を$\rho(x, y, z)$とおく.

微小な直方体(体積$V_0=\Delta x\Delta y\Delta z$)を考えて,
この体積要素が微小時間で増加する体積$\Delta v$を調べると,
\begin{align}
\Delta v
&=\rho(x+\Delta x, y, z)\{u_1(x+\Delta x, y, z)\Delta t\} \Delta y \Delta z -\rho(x, y, z)\{u_1(x, y, z)\Delta t\}\Delta y \Delta z \nonumber \\
&+\rho(x, y+\Delta y, z)\{u_2(x, y+\Delta y, z)\Delta t\} \Delta x \Delta z -\rho(x, y, z)\{u_2(x, y, z)\Delta t\}\Delta x \Delta z \nonumber \\
&+\rho(x, y, z+\Delta z)\{u_3(x, y, z+\Delta z)\Delta t\} \Delta x \Delta y -\rho(x, y, z)\{u_3(x, y, z)\Delta t\}\Delta x \Delta y
\end{align}
である.これを整理して,$\Delta x, \Delta y, \Delta z, \Delta t\rightarrow 0$
の極限を取ると,
\begin{align}
\frac{\partial\rho}{\partial t}=-~\divergence(\rho~{\bm u})
\end{align}
と分かる.(ここで$\divergence$はベクトル解析の発散である.)
これが一般的な連続の式である.

ここで,密度が急激に変化しないという仮定をおくと,
右辺の$\rho$を係数として外に出すことが出来る.
質量が保存することから密度と体積の関係も線形近似でき,
\begin{align}
\frac{\partial \Delta v}{\partial t}=V_0~\divergence {\bm u}
\end{align}
となる.

さらに,先ほど求めた体積と圧力の関係を用いると,
\begin{align}
\frac{\partial p}{\partial t}=-\gamma~\divergence {\bm u}
\end{align}
と変形できる.

\subsubsection{運動方程式}

流体の運動方程式は``ナビエ・ストークス方程式''であるが,
非常に単純化されたモデルであればニュートンの運動方程式でも同じ式が導ける.

$x$方向の運動について考えると,
\begin{align}
\rho\Delta x\Delta y\Delta z \frac{\partial u_x}{\partial t}
=-p(x+\Delta x, y, z)\Delta y\Delta z+p(x, y, z)\Delta y\Delta z
\end{align}
であり,極限を取ると
\begin{align}
\rho\frac{\partial u_x}{\partial t}=-\frac{\partial p}{\partial x}
\end{align}
である.$y, z$方向についても同様なので,
\begin{align}
\rho \frac{\partial u}{\partial t}=-\grad p
\end{align}
がいえる.

より一般的な``ナビエ・ストークス方程式''と比較すると,
対流項と拡散項が無視されていることが分かる.

\subsubsection{波動方程式}

ここまで求めてきた運動方程式と連続の式を連立させることで,
波動方程式を導出することが出来る.
\begin{align}
\begin{cases}
\displaystyle
\frac{\partial {\bm u}}{\partial t}=-\frac{1}{p}\grad p & (運動方程式) \\
\displaystyle
\frac{\partial p}{\partial t}=-\gamma~\divergence {\bm u} & (連続の式)
\end{cases}
\end{align}

上式の両辺に発散を,下式の両辺に時間微分を適用し,連立させることで,
波動方程式が導出できる.
\begin{align}
\frac{\partial^2 p}{\partial t^2}=\frac{1}{c^2} \nabla^2 p \\
ただし c=\sqrt{\frac{\gamma}{\rho}} は音速.
\end{align}
ここで$\nabla^2 p=\nabla\cdot \nabla p=\divergence \grad p$はラプラシアンと呼ばれる演算である.

\subsection{FDTD法による近似計算}

波動方程式に差分法を適用する場合,
2階微分が含まれることに注意する必要がある.

以下の波動方程式
\begin{align}
\frac{\partial^2 p}{\partial t^2} =c^2 \nabla^2 p
\end{align}
について,時間ステップ幅$\Delta t$,空間格子間隔$\Delta h$とおいて,
差分法を適用する.
\begin{align}
\begin{aligned}
&\frac{1}{c^2} \frac{p(x, y, z; t)-2p(x, y, z; t-\Delta t)+p(x, y, z; t-2\Delta t)}{(\Delta t)^2} \\
&=c^2\left\{\frac{p(x+\Delta x, y, z; t-\Delta t)-2p(x, y, z; t-\Delta t)+p(x-\Delta x, y, z; t-\Delta t)}{(\Delta h)^2} \right.\\
&+\frac{p(x, y+\Delta y, z; t-\Delta t)-2p(x, y, z; t-\Delta t)+p(x, y-\Delta y, z; t-\Delta t)}{(\Delta h)^2} \\
&\left.+\frac{p(x, y, z+\Delta z; t-\Delta t)-2p(x, y, z; t-\Delta t)+p(x, y, z-\Delta z; t-\Delta t)}{(\Delta h)^2}\right\}
\end{aligned}
\end{align}
整理すると,
\begin{align}
p(x, y, z; t)=\frac{(c\Delta t)^2}{(\delta h)^2}\{p(x+\Delta x, y, z; t-\Delta t)+\cdots\}+2p(x, y, z; t-\Delta t)-p(x, y, z; t-2\Delta t)
\end{align}
となり,時間ステップの更新式が得られた.

\begin{comment}

\section{テンプレート Template}

これはテストです。

This is a typeset template on \LaTeX.

\begin{quotation}
This is a quotation. 

これは引用文です。
\end{quotation}

Use \verb|\[\]| to show an expression:
\[a^2+b^2=c^2\]

The alternative is \verb|align*|:
\begin{align*}
a^2+b^2=c^2
\end{align*}

If expression numbers are needed, then use \verb|align|:
\begin{align}
a^2+b^2=c^2
\end{align}

\section{フォント Font}
Available font styles are:
\begin{itemize}
\item \textbf{Bold 太字}
\item \textit{Italic イタリック体}
\item \textrm{Roman ロマーン体}
\item \textsf{Sanserif サンセリフ体}
\item \textsc{Small Capital スモールキャピタル体}
\item \texttt{Typewriter タイプライタ体}
\item \textsl{Slant スラント体}
\item \emph{Emphasized 強調}
\end{itemize}

\section{ソースコード Source Code}
Source code is below:
\begin{lstlisting}[language=c]
int main()
{
    return 0;
}
\end{lstlisting}


\documentclass[unicode,12pt]{beamer}
\usepackage[no-math]{luatexja-fontspec}
\usepackage{graphicx}
\usepackage{tcolorbox}
\usepackage{xifthen}
\usepackage{tikz}
\usepackage{listings}
\usepackage{bm}

% Looks
\setbeamertemplate{navigation symbols}{}
\renewcommand{\baselinestretch}{1.25}
\setbeamersize{text margin left=15pt,text margin right=15pt}

% Fonts
\setsansfont[
    BoldFont=OpenSans-Bold,
    FontFace={sb}{n}{Font=OpenSans-Semibold},
]{OpenSans-Regular}
\setmainjfont[
    BoldFont=Gen Shin Gothic P Bold,
    YokoFeatures={JFM=prop},
    FontFace={sb}{n}{Font=Gen Shin Gothic P Medium},
]{Gen Shin Gothic P Normal}
\setmonofont{inconsolata}
\usepackage{newtxmath}
\usefonttheme{professionalfonts}
\setbeamerfont{structure}{series=\fontseries{sb}}
\setbeamerfont{alerted text}{series=\bfseries}
\setbeamerfont{frametitle}{size=\large,series=\bfseries}
\setbeamerfont{title}{size=\Large,series=\bfseries}
\setbeamerfont{subtitle}{size=\normalsize,series=\fontseries{sb}}
\setbeamerfont{author}{size=\normalsize,series=\fontseries{sb}}
\setbeamerfont{institute}{size=\small}

% Color
\definecolor{WhiteBlue}{HTML}{f9fdff}
\definecolor{LightBlue}{HTML}{dbf2ff}
\definecolor{Blue}{HTML}{2d5ae0}
\definecolor{DarkBlue}{HTML}{143447}

\definecolor{WhiteOrange}{HTML}{fffcf9}
\definecolor{LightOrange}{HTML}{ffeee0}
\definecolor{Orange}{HTML}{e33900}
\definecolor{DarkOrange}{HTML}{4f2604}

\definecolor{Purple}{HTML}{9400d3}
\definecolor{WhitePurple}{HTML}{faf0ff}
\definecolor{DarkPurple}{HTML}{441c58}

\definecolor{Green}{HTML}{009e73}
\definecolor{WhiteGreen}{HTML}{eefffa}
\definecolor{DarkGreen}{HTML}{144437}

\setbeamercolor{normal text}{fg=DarkGreen}
\setbeamercolor{structure}{fg=Green}
\setbeamercolor{alerted text}{fg=Purple}
\setbeamercolor{example text}{fg=Green}
\setbeamercolor{boxdefault}{fg=WhiteGreen,bg=Green}
\setbeamercolor{boxalerted}{fg=WhiteOrange,bg=Purple}
\setbeamercolor{section in toc shaded}{fg=DarkGreen}

% Footer
\setbeamertemplate{footline}[frame number]
\setbeamerfont{footline}{size=\normalsize}

% Framed Block
\defbeamertemplateparent{blocks}[framed]{block begin,block end}[1][]
{[#1]}
\defbeamertemplate{block begin}{framed}[1][]{
    \begin{tcolorbox}[
        colback=WhiteGreen,
        colframe=Green,
        colbacktitle=Green,
        coltitle=WhiteGreen,
        coltext=DarkGreen,
        sharp corners,
        boxrule=0.7pt,
        title=\fontseries{sb}\selectfont\insertblocktitle
    ]
    \usebeamerfont{block body}
}
\defbeamertemplate{block end}{framed}[1][]{\end{tcolorbox}}
\setbeamertemplate{blocks}[framed]
\defbeamertemplateparent{blocks alerted}[framed]{block alerted begin,block alerted end}[1][]
{[#1]}
\defbeamertemplate{block alerted begin}{framed}[1][]{
    \begin{tcolorbox}[
        colback=WhitePurple,
        colframe=Purple,
        colbacktitle=Purple,
        coltitle=WhitePurple,
        coltext=DarkPurple,
        sharp corners,
        boxrule=0.7pt,
        title=\bfseries\insertblocktitle
    ]
    \usebeamerfont{block alerted body}
}
\defbeamertemplate{block alerted end}{framed}[1][]{\end{tcolorbox}}
\setbeamertemplate{blocks alerted}[framed]

% Title Frame
\setbeamertemplate{frametitle}{
    \vspace{-1mm}
    \usebeamerfont{frametitle}\usebeamercolor[fg]{frametitle}\insertframetitle\par
    \vspace{-4.5mm}\hspace{-3mm}
    \begin{tikzpicture}
        \draw (0,0) -- (10,0);
    \end{tikzpicture}
}

% Title
\setbeamertemplate{title page}{
    \begin{flushright}
        {
            \usebeamerfont{title}\usebeamercolor[fg]{title}\inserttitle
            \vspace{-4mm}
            \begin{tikzpicture}
                \draw (0,0) -- (10,0);
            \end{tikzpicture}
        } \\
        \vspace{-4mm}
        {
            \usebeamerfont{subtitle}\usebeamercolor[fg]{subtitle}
            \ifthenelse{\isempty{\subtitle}}{}{\vspace{2mm}\insertsubtitle}
        } \\
        \vspace{8mm}
        {
            \usebeamerfont{author}\usebeamercolor[fg]{author}
            \ifthenelse{\isempty{\author}}{}{\insertauthor}
        } \\
        {
            \usebeamerfont{institute}\usebeamercolor[fg]{institute}
            \ifthenelse{\isempty{\institute}}{}{\vspace{1mm}\insertinstitute}
        } \\
        \vspace{8mm}
        {
            \usebeamerfont{date}\usebeamercolor[fg]{date}
            \ifthenelse{\isempty{\date}}{}{\insertdate}
        }
    \end{flushright}
}

% Listings
\lstset{
    language=C++,
    basicstyle=\ttfamily\footnotesize,
    keywordstyle=\bfseries\color{Green},
    stringstyle=\color{DarkOrange},
    commentstyle=\color{Purple},
    showstringspaces=false,
    breaklines=true,
    columns=[l]{fullflexible},
    lineskip=2pt,
}

% Itemize
\let\OLDitemize\itemize
\renewcommand{\itemize}{\OLDitemize\setlength{\itemindent}{-5pt}}
\let\OLDdescription\description
\renewcommand{\description}{\OLDdescription\setlength{\itemindent}{-20pt}}
\setbeamertemplate{itemize items}[circle]

% Biblatex
\usepackage[backend=bibtex,style=ieee]{biblatex}
\addbibresource{reference.bib}
\AtEveryCitekey{\iffootnote{\scriptsize}{}}
\setbeamertemplate{bibliography item}[text]

% Caption
\setbeamerfont{caption}{size=\footnotesize}
\setbeamertemplate{caption}[numbered]
\setbeamertemplate{caption label separator}{}
\setlength\abovecaptionskip{-5pt}
\renewcommand{\figurename}{Fig.}

% misc
\newtheorem{thm}{定理}
\newtheorem{lem}{補題}


\title{ノイズキャンセリング}
\author{TA~~~遠藤 亘}
\date{2014-11-18}

\begin{document}

\maketitle

\section{問題}

ある直方体の空間中に騒音源があり,
これによって発生する騒音を空間内部のある領域
において除去したい.
騒音源は周波数,位相,振幅が全て既知の正弦波を発生させるとし,
消音スピーカーを適宜配置することで,
特定の領域に発生する音波のエネルギーを最小化せよ.

\section{小問題}

\subsection{基本方程式の導出}

波動は,流体以外にも電磁波のような異なる物理現象においても,
波動方程式と呼ばれる一意な式によって表せる.

空気の``運動方程式''と,``連続の式''から,適切な線形近似を行い,
波動方程式を導出せよ.

\subsection{FDTD法の適用}

偏微分方程式を数値的に解く方法として,
基本的な手法として差分法が知られている.
特に,時間微分と空間微分の両方を差分近似する手法はFDTD法と呼ばれ,
波動方程式の最も基本的な数値解析手法である.

上で求めた波動方程式から,
FDTD法による数値計算に必要な時間ステップ毎の更新式を導出せよ.

数値シミュレーションをする上では,
更新式だけでなく,初期条件と境界条件も重要な要素である.
特に音波が壁で反射するような状況を考慮すると,
境界条件はどのように指定すればいいだろうか.

\subsection{FDTD法のプログラム}

FDTD法によって実際に数値計算を行うプログラムを記述し,
音波伝搬の様子をグラフ化せよ.

まずは,最も単純に,ある点の周りの6点の値を基に
差を計算するプログラムを記述して,
きちんと波が伝搬することを確認するとよい.
しかし,これでは実行性能に問題があるので,
改善手法について検討せよ.

\subsection{エネルギー最小化}

FDTD法による音波伝搬シミュレータを用いて,
スピーカーのパラメータを複数試し,
音を最小化出来るパラメータを
求めるプログラムを作成せよ.

\section{発展課題}

\subsection{格子モデルの改良}

同じFDTD法によるシミュレーションであっても,
格子点をずらしたスタッガード格子によって、
シミュレーションの誤差を低減できることがあること
が知られている.

先ほどまでの通常格子に加えて,
スタッガード格子を利用したプログラムも作成し,
両者の結果を比較せよ.

\subsection{PyAudioを試す}

Pythonには,波形を実際に音声として再生するPyAudioというモジュールがある.
これまでシミュレーションした結果について,
空間のどこか1点について実際の音として再生してみよ.

余裕があれば,ドップラー効果のような
波動現象について,実際にシミュレーションで再現できているかどうか確認できるとよい.


\subsection{任意波形への拡張 (やや難しい)}

これまでは騒音を正弦波であると仮定してきたが,
任意波形に拡張することを考えてみよ.
簡単のため,消音すべき場所は1点のみでよく,
消音スピーカーも1つだけとする.


\subsection{適応的なノイズキャンセリング (難しい)}

マイクロフォンで集音して,その波形を基に
スピーカーから音を出力すれば,
入力波形が未知であったとしても
ノイズキャンセリングを行えると期待される.
未知波形のノイズキャンセリングについて検討せよ.


\subsection{非線形音響現象 (おそらくとても難しい)}

空気は実際には完全に線形な振る舞いをするわけではなく,
僅かではあるが非線形な成分が含まれている.
このような非線形現象を応用した例として,
パラメトリックアレイスピーカー等が知られている.

非線形な音波伝搬を解析するには,
通常は線形近似してしまう式をより厳密に計算する必要がある.
非線形現象の解析には,
音響分野にも最近応用され始めたCIP法等,
いくつかの数値計算手法が存在する(らしい).


\section{解説}


\subsection{波動方程式の導出}

\subsubsection{体積と圧力の関係}

ある体積要素について,微小時間における体積$V$と圧力$P$の変化の関係
について見ていく.

まず,微小時間の変化が``断熱過程''であるという仮定をおく.
すなわち,微小時間における体積要素内外の熱のやり取りは無いと仮定する.
この時,
\begin{align}
PV^{\gamma}={\rm const.}
\end{align}
が成り立つ.ただし$\gamma$は比熱比である.
ここで,圧力が$P_0\rightarrow P_0+\Delta p$,体積が$V_0\rightarrow V_0+\Delta v$と変化するなら,
\begin{align}
P_0V_0^\gamma=(P_0+\Delta p)(V_0+\Delta v)^\gamma
\end{align}
である.2次の微小項を取り除くことで,
\begin{align}
\frac{\Delta p}{P_0}=-\gamma \frac{\Delta v}{V_0}
\end{align}
が導かれる.これは,力 (圧力)と 幾何的距離 (体積)の比例関係を示しており,
フックの法則の一種である.
つまり,気体が弾性体として振る舞うことを示している.

\subsubsection{連続の式}

流体における質量保存の法則は,連続の式と呼ばれる.

ここから,ある点$(x, y, z)$の速度を${\bm u}(x, y, z)=(u_1, u_2, u_3)$,
密度を$\rho(x, y, z)$とおく.

微小な直方体(体積$V_0=\Delta x\Delta y\Delta z$)を考えて,
この体積要素が微小時間で増加する体積$\Delta v$を調べると,
\begin{align}
\Delta v
&=\rho(x+\Delta x, y, z)\{u_1(x+\Delta x, y, z)\Delta t\} \Delta y \Delta z -\rho(x, y, z)\{u_1(x, y, z)\Delta t\}\Delta y \Delta z \nonumber \\
&+\rho(x, y+\Delta y, z)\{u_2(x, y+\Delta y, z)\Delta t\} \Delta x \Delta z -\rho(x, y, z)\{u_2(x, y, z)\Delta t\}\Delta x \Delta z \nonumber \\
&+\rho(x, y, z+\Delta z)\{u_3(x, y, z+\Delta z)\Delta t\} \Delta x \Delta y -\rho(x, y, z)\{u_3(x, y, z)\Delta t\}\Delta x \Delta y
\end{align}
である.これを整理して,$\Delta x, \Delta y, \Delta z, \Delta t\rightarrow 0$
の極限を取ると,
\begin{align}
\frac{\partial\rho}{\partial t}=-~\divergence(\rho~{\bm u})
\end{align}
と分かる.(ここで$\divergence$はベクトル解析の発散である.)
これが一般的な連続の式である.

ここで,密度が急激に変化しないという仮定をおくと,
右辺の$\rho$を係数として外に出すことが出来る.
質量が保存することから密度と体積の関係も線形近似でき,
\begin{align}
\frac{\partial \Delta v}{\partial t}=V_0~\divergence {\bm u}
\end{align}
となる.

さらに,先ほど求めた体積と圧力の関係を用いると,
\begin{align}
\frac{\partial p}{\partial t}=-\gamma~\divergence {\bm u}
\end{align}
と変形できる.

\subsubsection{運動方程式}

流体の運動方程式は``ナビエ・ストークス方程式''であるが,
非常に単純化されたモデルであればニュートンの運動方程式でも同じ式が導ける.

$x$方向の運動について考えると,
\begin{align}
\rho\Delta x\Delta y\Delta z \frac{\partial u_x}{\partial t}
=-p(x+\Delta x, y, z)\Delta y\Delta z+p(x, y, z)\Delta y\Delta z
\end{align}
であり,極限を取ると
\begin{align}
\rho\frac{\partial u_x}{\partial t}=-\frac{\partial p}{\partial x}
\end{align}
である.$y, z$方向についても同様なので,
\begin{align}
\rho \frac{\partial u}{\partial t}=-\grad p
\end{align}
がいえる.

より一般的な``ナビエ・ストークス方程式''と比較すると,
対流項と拡散項が無視されていることが分かる.

\subsubsection{波動方程式}

ここまで求めてきた運動方程式と連続の式を連立させることで,
波動方程式を導出することが出来る.
\begin{align}
\begin{cases}
\displaystyle
\frac{\partial {\bm u}}{\partial t}=-\frac{1}{p}\grad p & (運動方程式) \\
\displaystyle
\frac{\partial p}{\partial t}=-\gamma~\divergence {\bm u} & (連続の式)
\end{cases}
\end{align}

上式の両辺に発散を,下式の両辺に時間微分を適用し,連立させることで,
波動方程式が導出できる.
\begin{align}
\frac{\partial^2 p}{\partial t^2}=\frac{1}{c^2} \nabla^2 p \\
ただし c=\sqrt{\frac{\gamma}{\rho}} は音速.
\end{align}
ここで$\nabla^2 p=\nabla\cdot \nabla p=\divergence \grad p$はラプラシアンと呼ばれる演算である.

\subsection{FDTD法による近似計算}

波動方程式に差分法を適用する場合,
2階微分が含まれることに注意する必要がある.

以下の波動方程式
\begin{align}
\frac{\partial^2 p}{\partial t^2} =c^2 \nabla^2 p
\end{align}
について,時間ステップ幅$\Delta t$,空間格子間隔$\Delta h$とおいて,
差分法を適用する.
\begin{align}
\begin{aligned}
&\frac{1}{c^2} \frac{p(x, y, z; t)-2p(x, y, z; t-\Delta t)+p(x, y, z; t-2\Delta t)}{(\Delta t)^2} \\
&=c^2\left\{\frac{p(x+\Delta x, y, z; t-\Delta t)-2p(x, y, z; t-\Delta t)+p(x-\Delta x, y, z; t-\Delta t)}{(\Delta h)^2} \right.\\
&+\frac{p(x, y+\Delta y, z; t-\Delta t)-2p(x, y, z; t-\Delta t)+p(x, y-\Delta y, z; t-\Delta t)}{(\Delta h)^2} \\
&\left.+\frac{p(x, y, z+\Delta z; t-\Delta t)-2p(x, y, z; t-\Delta t)+p(x, y, z-\Delta z; t-\Delta t)}{(\Delta h)^2}\right\}
\end{aligned}
\end{align}
整理すると,
\begin{align}
p(x, y, z; t)=\frac{(c\Delta t)^2}{(\delta h)^2}\{p(x+\Delta x, y, z; t-\Delta t)+\cdots\}+2p(x, y, z; t-\Delta t)-p(x, y, z; t-2\Delta t)
\end{align}
となり,時間ステップの更新式が得られた.

\begin{comment}

\section{テンプレート Template}

これはテストです。

This is a typeset template on \LaTeX.

\begin{quotation}
This is a quotation. 

これは引用文です。
\end{quotation}

Use \verb|\[\]| to show an expression:
\[a^2+b^2=c^2\]

The alternative is \verb|align*|:
\begin{align*}
a^2+b^2=c^2
\end{align*}

If expression numbers are needed, then use \verb|align|:
\begin{align}
a^2+b^2=c^2
\end{align}

\section{フォント Font}
Available font styles are:
\begin{itemize}
\item \textbf{Bold 太字}
\item \textit{Italic イタリック体}
\item \textrm{Roman ロマーン体}
\item \textsf{Sanserif サンセリフ体}
\item \textsc{Small Capital スモールキャピタル体}
\item \texttt{Typewriter タイプライタ体}
\item \textsl{Slant スラント体}
\item \emph{Emphasized 強調}
\end{itemize}

\section{ソースコード Source Code}
Source code is below:
\begin{lstlisting}[language=c]
int main()
{
    return 0;
}
\end{lstlisting}


\documentclass[unicode,12pt]{beamer}
\usepackage[no-math]{luatexja-fontspec}
\usepackage{graphicx}
\usepackage{tcolorbox}
\usepackage{xifthen}
\usepackage{tikz}
\usepackage{listings}
\usepackage{bm}

% Looks
\setbeamertemplate{navigation symbols}{}
\renewcommand{\baselinestretch}{1.25}
\setbeamersize{text margin left=15pt,text margin right=15pt}

% Fonts
\setsansfont[
    BoldFont=OpenSans-Bold,
    FontFace={sb}{n}{Font=OpenSans-Semibold},
]{OpenSans-Regular}
\setmainjfont[
    BoldFont=Gen Shin Gothic P Bold,
    YokoFeatures={JFM=prop},
    FontFace={sb}{n}{Font=Gen Shin Gothic P Medium},
]{Gen Shin Gothic P Normal}
\setmonofont{inconsolata}
\usepackage{newtxmath}
\usefonttheme{professionalfonts}
\setbeamerfont{structure}{series=\fontseries{sb}}
\setbeamerfont{alerted text}{series=\bfseries}
\setbeamerfont{frametitle}{size=\large,series=\bfseries}
\setbeamerfont{title}{size=\Large,series=\bfseries}
\setbeamerfont{subtitle}{size=\normalsize,series=\fontseries{sb}}
\setbeamerfont{author}{size=\normalsize,series=\fontseries{sb}}
\setbeamerfont{institute}{size=\small}

% Color
\definecolor{WhiteBlue}{HTML}{f9fdff}
\definecolor{LightBlue}{HTML}{dbf2ff}
\definecolor{Blue}{HTML}{2d5ae0}
\definecolor{DarkBlue}{HTML}{143447}

\definecolor{WhiteOrange}{HTML}{fffcf9}
\definecolor{LightOrange}{HTML}{ffeee0}
\definecolor{Orange}{HTML}{e33900}
\definecolor{DarkOrange}{HTML}{4f2604}

\definecolor{Purple}{HTML}{9400d3}
\definecolor{WhitePurple}{HTML}{faf0ff}
\definecolor{DarkPurple}{HTML}{441c58}

\definecolor{Green}{HTML}{009e73}
\definecolor{WhiteGreen}{HTML}{eefffa}
\definecolor{DarkGreen}{HTML}{144437}

\setbeamercolor{normal text}{fg=DarkGreen}
\setbeamercolor{structure}{fg=Green}
\setbeamercolor{alerted text}{fg=Purple}
\setbeamercolor{example text}{fg=Green}
\setbeamercolor{boxdefault}{fg=WhiteGreen,bg=Green}
\setbeamercolor{boxalerted}{fg=WhiteOrange,bg=Purple}
\setbeamercolor{section in toc shaded}{fg=DarkGreen}

% Footer
\setbeamertemplate{footline}[frame number]
\setbeamerfont{footline}{size=\normalsize}

% Framed Block
\defbeamertemplateparent{blocks}[framed]{block begin,block end}[1][]
{[#1]}
\defbeamertemplate{block begin}{framed}[1][]{
    \begin{tcolorbox}[
        colback=WhiteGreen,
        colframe=Green,
        colbacktitle=Green,
        coltitle=WhiteGreen,
        coltext=DarkGreen,
        sharp corners,
        boxrule=0.7pt,
        title=\fontseries{sb}\selectfont\insertblocktitle
    ]
    \usebeamerfont{block body}
}
\defbeamertemplate{block end}{framed}[1][]{\end{tcolorbox}}
\setbeamertemplate{blocks}[framed]
\defbeamertemplateparent{blocks alerted}[framed]{block alerted begin,block alerted end}[1][]
{[#1]}
\defbeamertemplate{block alerted begin}{framed}[1][]{
    \begin{tcolorbox}[
        colback=WhitePurple,
        colframe=Purple,
        colbacktitle=Purple,
        coltitle=WhitePurple,
        coltext=DarkPurple,
        sharp corners,
        boxrule=0.7pt,
        title=\bfseries\insertblocktitle
    ]
    \usebeamerfont{block alerted body}
}
\defbeamertemplate{block alerted end}{framed}[1][]{\end{tcolorbox}}
\setbeamertemplate{blocks alerted}[framed]

% Title Frame
\setbeamertemplate{frametitle}{
    \vspace{-1mm}
    \usebeamerfont{frametitle}\usebeamercolor[fg]{frametitle}\insertframetitle\par
    \vspace{-4.5mm}\hspace{-3mm}
    \begin{tikzpicture}
        \draw (0,0) -- (10,0);
    \end{tikzpicture}
}

% Title
\setbeamertemplate{title page}{
    \begin{flushright}
        {
            \usebeamerfont{title}\usebeamercolor[fg]{title}\inserttitle
            \vspace{-4mm}
            \begin{tikzpicture}
                \draw (0,0) -- (10,0);
            \end{tikzpicture}
        } \\
        \vspace{-4mm}
        {
            \usebeamerfont{subtitle}\usebeamercolor[fg]{subtitle}
            \ifthenelse{\isempty{\subtitle}}{}{\vspace{2mm}\insertsubtitle}
        } \\
        \vspace{8mm}
        {
            \usebeamerfont{author}\usebeamercolor[fg]{author}
            \ifthenelse{\isempty{\author}}{}{\insertauthor}
        } \\
        {
            \usebeamerfont{institute}\usebeamercolor[fg]{institute}
            \ifthenelse{\isempty{\institute}}{}{\vspace{1mm}\insertinstitute}
        } \\
        \vspace{8mm}
        {
            \usebeamerfont{date}\usebeamercolor[fg]{date}
            \ifthenelse{\isempty{\date}}{}{\insertdate}
        }
    \end{flushright}
}

% Listings
\lstset{
    language=C++,
    basicstyle=\ttfamily\footnotesize,
    keywordstyle=\bfseries\color{Green},
    stringstyle=\color{DarkOrange},
    commentstyle=\color{Purple},
    showstringspaces=false,
    breaklines=true,
    columns=[l]{fullflexible},
    lineskip=2pt,
}

% Itemize
\let\OLDitemize\itemize
\renewcommand{\itemize}{\OLDitemize\setlength{\itemindent}{-5pt}}
\let\OLDdescription\description
\renewcommand{\description}{\OLDdescription\setlength{\itemindent}{-20pt}}
\setbeamertemplate{itemize items}[circle]

% Biblatex
\usepackage[backend=bibtex,style=ieee]{biblatex}
\addbibresource{reference.bib}
\AtEveryCitekey{\iffootnote{\scriptsize}{}}
\setbeamertemplate{bibliography item}[text]

% Caption
\setbeamerfont{caption}{size=\footnotesize}
\setbeamertemplate{caption}[numbered]
\setbeamertemplate{caption label separator}{}
\setlength\abovecaptionskip{-5pt}
\renewcommand{\figurename}{Fig.}

% misc
\newtheorem{thm}{定理}
\newtheorem{lem}{補題}


\title{ノイズキャンセリング}
\author{TA~~~遠藤 亘}
\date{2014-11-18}

\begin{document}

\maketitle

\section{問題}

ある直方体の空間中に騒音源があり,
これによって発生する騒音を空間内部のある領域
において除去したい.
騒音源は周波数,位相,振幅が全て既知の正弦波を発生させるとし,
消音スピーカーを適宜配置することで,
特定の領域に発生する音波のエネルギーを最小化せよ.

\section{小問題}

\subsection{基本方程式の導出}

波動は,流体以外にも電磁波のような異なる物理現象においても,
波動方程式と呼ばれる一意な式によって表せる.

空気の``運動方程式''と,``連続の式''から,適切な線形近似を行い,
波動方程式を導出せよ.

\subsection{FDTD法の適用}

偏微分方程式を数値的に解く方法として,
基本的な手法として差分法が知られている.
特に,時間微分と空間微分の両方を差分近似する手法はFDTD法と呼ばれ,
波動方程式の最も基本的な数値解析手法である.

上で求めた波動方程式から,
FDTD法による数値計算に必要な時間ステップ毎の更新式を導出せよ.

数値シミュレーションをする上では,
更新式だけでなく,初期条件と境界条件も重要な要素である.
特に音波が壁で反射するような状況を考慮すると,
境界条件はどのように指定すればいいだろうか.

\subsection{FDTD法のプログラム}

FDTD法によって実際に数値計算を行うプログラムを記述し,
音波伝搬の様子をグラフ化せよ.

まずは,最も単純に,ある点の周りの6点の値を基に
差を計算するプログラムを記述して,
きちんと波が伝搬することを確認するとよい.
しかし,これでは実行性能に問題があるので,
改善手法について検討せよ.

\subsection{エネルギー最小化}

FDTD法による音波伝搬シミュレータを用いて,
スピーカーのパラメータを複数試し,
音を最小化出来るパラメータを
求めるプログラムを作成せよ.

\section{発展課題}

\subsection{格子モデルの改良}

同じFDTD法によるシミュレーションであっても,
格子点をずらしたスタッガード格子によって、
シミュレーションの誤差を低減できることがあること
が知られている.

先ほどまでの通常格子に加えて,
スタッガード格子を利用したプログラムも作成し,
両者の結果を比較せよ.

\subsection{PyAudioを試す}

Pythonには,波形を実際に音声として再生するPyAudioというモジュールがある.
これまでシミュレーションした結果について,
空間のどこか1点について実際の音として再生してみよ.

余裕があれば,ドップラー効果のような
波動現象について,実際にシミュレーションで再現できているかどうか確認できるとよい.


\subsection{任意波形への拡張 (やや難しい)}

これまでは騒音を正弦波であると仮定してきたが,
任意波形に拡張することを考えてみよ.
簡単のため,消音すべき場所は1点のみでよく,
消音スピーカーも1つだけとする.


\subsection{適応的なノイズキャンセリング (難しい)}

マイクロフォンで集音して,その波形を基に
スピーカーから音を出力すれば,
入力波形が未知であったとしても
ノイズキャンセリングを行えると期待される.
未知波形のノイズキャンセリングについて検討せよ.


\subsection{非線形音響現象 (おそらくとても難しい)}

空気は実際には完全に線形な振る舞いをするわけではなく,
僅かではあるが非線形な成分が含まれている.
このような非線形現象を応用した例として,
パラメトリックアレイスピーカー等が知られている.

非線形な音波伝搬を解析するには,
通常は線形近似してしまう式をより厳密に計算する必要がある.
非線形現象の解析には,
音響分野にも最近応用され始めたCIP法等,
いくつかの数値計算手法が存在する(らしい).


\section{解説}


\subsection{波動方程式の導出}

\subsubsection{体積と圧力の関係}

ある体積要素について,微小時間における体積$V$と圧力$P$の変化の関係
について見ていく.

まず,微小時間の変化が``断熱過程''であるという仮定をおく.
すなわち,微小時間における体積要素内外の熱のやり取りは無いと仮定する.
この時,
\begin{align}
PV^{\gamma}={\rm const.}
\end{align}
が成り立つ.ただし$\gamma$は比熱比である.
ここで,圧力が$P_0\rightarrow P_0+\Delta p$,体積が$V_0\rightarrow V_0+\Delta v$と変化するなら,
\begin{align}
P_0V_0^\gamma=(P_0+\Delta p)(V_0+\Delta v)^\gamma
\end{align}
である.2次の微小項を取り除くことで,
\begin{align}
\frac{\Delta p}{P_0}=-\gamma \frac{\Delta v}{V_0}
\end{align}
が導かれる.これは,力 (圧力)と 幾何的距離 (体積)の比例関係を示しており,
フックの法則の一種である.
つまり,気体が弾性体として振る舞うことを示している.

\subsubsection{連続の式}

流体における質量保存の法則は,連続の式と呼ばれる.

ここから,ある点$(x, y, z)$の速度を${\bm u}(x, y, z)=(u_1, u_2, u_3)$,
密度を$\rho(x, y, z)$とおく.

微小な直方体(体積$V_0=\Delta x\Delta y\Delta z$)を考えて,
この体積要素が微小時間で増加する体積$\Delta v$を調べると,
\begin{align}
\Delta v
&=\rho(x+\Delta x, y, z)\{u_1(x+\Delta x, y, z)\Delta t\} \Delta y \Delta z -\rho(x, y, z)\{u_1(x, y, z)\Delta t\}\Delta y \Delta z \nonumber \\
&+\rho(x, y+\Delta y, z)\{u_2(x, y+\Delta y, z)\Delta t\} \Delta x \Delta z -\rho(x, y, z)\{u_2(x, y, z)\Delta t\}\Delta x \Delta z \nonumber \\
&+\rho(x, y, z+\Delta z)\{u_3(x, y, z+\Delta z)\Delta t\} \Delta x \Delta y -\rho(x, y, z)\{u_3(x, y, z)\Delta t\}\Delta x \Delta y
\end{align}
である.これを整理して,$\Delta x, \Delta y, \Delta z, \Delta t\rightarrow 0$
の極限を取ると,
\begin{align}
\frac{\partial\rho}{\partial t}=-~\divergence(\rho~{\bm u})
\end{align}
と分かる.(ここで$\divergence$はベクトル解析の発散である.)
これが一般的な連続の式である.

ここで,密度が急激に変化しないという仮定をおくと,
右辺の$\rho$を係数として外に出すことが出来る.
質量が保存することから密度と体積の関係も線形近似でき,
\begin{align}
\frac{\partial \Delta v}{\partial t}=V_0~\divergence {\bm u}
\end{align}
となる.

さらに,先ほど求めた体積と圧力の関係を用いると,
\begin{align}
\frac{\partial p}{\partial t}=-\gamma~\divergence {\bm u}
\end{align}
と変形できる.

\subsubsection{運動方程式}

流体の運動方程式は``ナビエ・ストークス方程式''であるが,
非常に単純化されたモデルであればニュートンの運動方程式でも同じ式が導ける.

$x$方向の運動について考えると,
\begin{align}
\rho\Delta x\Delta y\Delta z \frac{\partial u_x}{\partial t}
=-p(x+\Delta x, y, z)\Delta y\Delta z+p(x, y, z)\Delta y\Delta z
\end{align}
であり,極限を取ると
\begin{align}
\rho\frac{\partial u_x}{\partial t}=-\frac{\partial p}{\partial x}
\end{align}
である.$y, z$方向についても同様なので,
\begin{align}
\rho \frac{\partial u}{\partial t}=-\grad p
\end{align}
がいえる.

より一般的な``ナビエ・ストークス方程式''と比較すると,
対流項と拡散項が無視されていることが分かる.

\subsubsection{波動方程式}

ここまで求めてきた運動方程式と連続の式を連立させることで,
波動方程式を導出することが出来る.
\begin{align}
\begin{cases}
\displaystyle
\frac{\partial {\bm u}}{\partial t}=-\frac{1}{p}\grad p & (運動方程式) \\
\displaystyle
\frac{\partial p}{\partial t}=-\gamma~\divergence {\bm u} & (連続の式)
\end{cases}
\end{align}

上式の両辺に発散を,下式の両辺に時間微分を適用し,連立させることで,
波動方程式が導出できる.
\begin{align}
\frac{\partial^2 p}{\partial t^2}=\frac{1}{c^2} \nabla^2 p \\
ただし c=\sqrt{\frac{\gamma}{\rho}} は音速.
\end{align}
ここで$\nabla^2 p=\nabla\cdot \nabla p=\divergence \grad p$はラプラシアンと呼ばれる演算である.

\subsection{FDTD法による近似計算}

波動方程式に差分法を適用する場合,
2階微分が含まれることに注意する必要がある.

以下の波動方程式
\begin{align}
\frac{\partial^2 p}{\partial t^2} =c^2 \nabla^2 p
\end{align}
について,時間ステップ幅$\Delta t$,空間格子間隔$\Delta h$とおいて,
差分法を適用する.
\begin{align}
\begin{aligned}
&\frac{1}{c^2} \frac{p(x, y, z; t)-2p(x, y, z; t-\Delta t)+p(x, y, z; t-2\Delta t)}{(\Delta t)^2} \\
&=c^2\left\{\frac{p(x+\Delta x, y, z; t-\Delta t)-2p(x, y, z; t-\Delta t)+p(x-\Delta x, y, z; t-\Delta t)}{(\Delta h)^2} \right.\\
&+\frac{p(x, y+\Delta y, z; t-\Delta t)-2p(x, y, z; t-\Delta t)+p(x, y-\Delta y, z; t-\Delta t)}{(\Delta h)^2} \\
&\left.+\frac{p(x, y, z+\Delta z; t-\Delta t)-2p(x, y, z; t-\Delta t)+p(x, y, z-\Delta z; t-\Delta t)}{(\Delta h)^2}\right\}
\end{aligned}
\end{align}
整理すると,
\begin{align}
p(x, y, z; t)=\frac{(c\Delta t)^2}{(\delta h)^2}\{p(x+\Delta x, y, z; t-\Delta t)+\cdots\}+2p(x, y, z; t-\Delta t)-p(x, y, z; t-2\Delta t)
\end{align}
となり,時間ステップの更新式が得られた.

\begin{comment}

\section{テンプレート Template}

これはテストです。

This is a typeset template on \LaTeX.

\begin{quotation}
This is a quotation. 

これは引用文です。
\end{quotation}

Use \verb|\[\]| to show an expression:
\[a^2+b^2=c^2\]

The alternative is \verb|align*|:
\begin{align*}
a^2+b^2=c^2
\end{align*}

If expression numbers are needed, then use \verb|align|:
\begin{align}
a^2+b^2=c^2
\end{align}

\section{フォント Font}
Available font styles are:
\begin{itemize}
\item \textbf{Bold 太字}
\item \textit{Italic イタリック体}
\item \textrm{Roman ロマーン体}
\item \textsf{Sanserif サンセリフ体}
\item \textsc{Small Capital スモールキャピタル体}
\item \texttt{Typewriter タイプライタ体}
\item \textsl{Slant スラント体}
\item \emph{Emphasized 強調}
\end{itemize}

\section{ソースコード Source Code}
Source code is below:
\begin{lstlisting}[language=c]
int main()
{
    return 0;
}
\end{lstlisting}


\include{header}

\title{ノイズキャンセリング}
\author{TA~~~遠藤 亘}
\date{2014-11-18}

\begin{document}

\maketitle

\section{問題}

ある直方体の空間中に騒音源があり,
これによって発生する騒音を空間内部のある領域
において除去したい.
騒音源は周波数,位相,振幅が全て既知の正弦波を発生させるとし,
消音スピーカーを適宜配置することで,
特定の領域に発生する音波のエネルギーを最小化せよ.

\section{小問題}

\subsection{基本方程式の導出}

波動は,流体以外にも電磁波のような異なる物理現象においても,
波動方程式と呼ばれる一意な式によって表せる.

空気の``運動方程式''と,``連続の式''から,適切な線形近似を行い,
波動方程式を導出せよ.

\subsection{FDTD法の適用}

偏微分方程式を数値的に解く方法として,
基本的な手法として差分法が知られている.
特に,時間微分と空間微分の両方を差分近似する手法はFDTD法と呼ばれ,
波動方程式の最も基本的な数値解析手法である.

上で求めた波動方程式から,
FDTD法による数値計算に必要な時間ステップ毎の更新式を導出せよ.

数値シミュレーションをする上では,
更新式だけでなく,初期条件と境界条件も重要な要素である.
特に音波が壁で反射するような状況を考慮すると,
境界条件はどのように指定すればいいだろうか.

\subsection{FDTD法のプログラム}

FDTD法によって実際に数値計算を行うプログラムを記述し,
音波伝搬の様子をグラフ化せよ.

まずは,最も単純に,ある点の周りの6点の値を基に
差を計算するプログラムを記述して,
きちんと波が伝搬することを確認するとよい.
しかし,これでは実行性能に問題があるので,
改善手法について検討せよ.

\subsection{エネルギー最小化}

FDTD法による音波伝搬シミュレータを用いて,
スピーカーのパラメータを複数試し,
音を最小化出来るパラメータを
求めるプログラムを作成せよ.

\section{発展課題}

\subsection{格子モデルの改良}

同じFDTD法によるシミュレーションであっても,
格子点をずらしたスタッガード格子によって、
シミュレーションの誤差を低減できることがあること
が知られている.

先ほどまでの通常格子に加えて,
スタッガード格子を利用したプログラムも作成し,
両者の結果を比較せよ.

\subsection{PyAudioを試す}

Pythonには,波形を実際に音声として再生するPyAudioというモジュールがある.
これまでシミュレーションした結果について,
空間のどこか1点について実際の音として再生してみよ.

余裕があれば,ドップラー効果のような
波動現象について,実際にシミュレーションで再現できているかどうか確認できるとよい.


\subsection{任意波形への拡張 (やや難しい)}

これまでは騒音を正弦波であると仮定してきたが,
任意波形に拡張することを考えてみよ.
簡単のため,消音すべき場所は1点のみでよく,
消音スピーカーも1つだけとする.


\subsection{適応的なノイズキャンセリング (難しい)}

マイクロフォンで集音して,その波形を基に
スピーカーから音を出力すれば,
入力波形が未知であったとしても
ノイズキャンセリングを行えると期待される.
未知波形のノイズキャンセリングについて検討せよ.


\subsection{非線形音響現象 (おそらくとても難しい)}

空気は実際には完全に線形な振る舞いをするわけではなく,
僅かではあるが非線形な成分が含まれている.
このような非線形現象を応用した例として,
パラメトリックアレイスピーカー等が知られている.

非線形な音波伝搬を解析するには,
通常は線形近似してしまう式をより厳密に計算する必要がある.
非線形現象の解析には,
音響分野にも最近応用され始めたCIP法等,
いくつかの数値計算手法が存在する(らしい).


\section{解説}


\subsection{波動方程式の導出}

\subsubsection{体積と圧力の関係}

ある体積要素について,微小時間における体積$V$と圧力$P$の変化の関係
について見ていく.

まず,微小時間の変化が``断熱過程''であるという仮定をおく.
すなわち,微小時間における体積要素内外の熱のやり取りは無いと仮定する.
この時,
\begin{align}
PV^{\gamma}={\rm const.}
\end{align}
が成り立つ.ただし$\gamma$は比熱比である.
ここで,圧力が$P_0\rightarrow P_0+\Delta p$,体積が$V_0\rightarrow V_0+\Delta v$と変化するなら,
\begin{align}
P_0V_0^\gamma=(P_0+\Delta p)(V_0+\Delta v)^\gamma
\end{align}
である.2次の微小項を取り除くことで,
\begin{align}
\frac{\Delta p}{P_0}=-\gamma \frac{\Delta v}{V_0}
\end{align}
が導かれる.これは,力 (圧力)と 幾何的距離 (体積)の比例関係を示しており,
フックの法則の一種である.
つまり,気体が弾性体として振る舞うことを示している.

\subsubsection{連続の式}

流体における質量保存の法則は,連続の式と呼ばれる.

ここから,ある点$(x, y, z)$の速度を${\bm u}(x, y, z)=(u_1, u_2, u_3)$,
密度を$\rho(x, y, z)$とおく.

微小な直方体(体積$V_0=\Delta x\Delta y\Delta z$)を考えて,
この体積要素が微小時間で増加する体積$\Delta v$を調べると,
\begin{align}
\Delta v
&=\rho(x+\Delta x, y, z)\{u_1(x+\Delta x, y, z)\Delta t\} \Delta y \Delta z -\rho(x, y, z)\{u_1(x, y, z)\Delta t\}\Delta y \Delta z \nonumber \\
&+\rho(x, y+\Delta y, z)\{u_2(x, y+\Delta y, z)\Delta t\} \Delta x \Delta z -\rho(x, y, z)\{u_2(x, y, z)\Delta t\}\Delta x \Delta z \nonumber \\
&+\rho(x, y, z+\Delta z)\{u_3(x, y, z+\Delta z)\Delta t\} \Delta x \Delta y -\rho(x, y, z)\{u_3(x, y, z)\Delta t\}\Delta x \Delta y
\end{align}
である.これを整理して,$\Delta x, \Delta y, \Delta z, \Delta t\rightarrow 0$
の極限を取ると,
\begin{align}
\frac{\partial\rho}{\partial t}=-~\divergence(\rho~{\bm u})
\end{align}
と分かる.(ここで$\divergence$はベクトル解析の発散である.)
これが一般的な連続の式である.

ここで,密度が急激に変化しないという仮定をおくと,
右辺の$\rho$を係数として外に出すことが出来る.
質量が保存することから密度と体積の関係も線形近似でき,
\begin{align}
\frac{\partial \Delta v}{\partial t}=V_0~\divergence {\bm u}
\end{align}
となる.

さらに,先ほど求めた体積と圧力の関係を用いると,
\begin{align}
\frac{\partial p}{\partial t}=-\gamma~\divergence {\bm u}
\end{align}
と変形できる.

\subsubsection{運動方程式}

流体の運動方程式は``ナビエ・ストークス方程式''であるが,
非常に単純化されたモデルであればニュートンの運動方程式でも同じ式が導ける.

$x$方向の運動について考えると,
\begin{align}
\rho\Delta x\Delta y\Delta z \frac{\partial u_x}{\partial t}
=-p(x+\Delta x, y, z)\Delta y\Delta z+p(x, y, z)\Delta y\Delta z
\end{align}
であり,極限を取ると
\begin{align}
\rho\frac{\partial u_x}{\partial t}=-\frac{\partial p}{\partial x}
\end{align}
である.$y, z$方向についても同様なので,
\begin{align}
\rho \frac{\partial u}{\partial t}=-\grad p
\end{align}
がいえる.

より一般的な``ナビエ・ストークス方程式''と比較すると,
対流項と拡散項が無視されていることが分かる.

\subsubsection{波動方程式}

ここまで求めてきた運動方程式と連続の式を連立させることで,
波動方程式を導出することが出来る.
\begin{align}
\begin{cases}
\displaystyle
\frac{\partial {\bm u}}{\partial t}=-\frac{1}{p}\grad p & (運動方程式) \\
\displaystyle
\frac{\partial p}{\partial t}=-\gamma~\divergence {\bm u} & (連続の式)
\end{cases}
\end{align}

上式の両辺に発散を,下式の両辺に時間微分を適用し,連立させることで,
波動方程式が導出できる.
\begin{align}
\frac{\partial^2 p}{\partial t^2}=\frac{1}{c^2} \nabla^2 p \\
ただし c=\sqrt{\frac{\gamma}{\rho}} は音速.
\end{align}
ここで$\nabla^2 p=\nabla\cdot \nabla p=\divergence \grad p$はラプラシアンと呼ばれる演算である.

\subsection{FDTD法による近似計算}

波動方程式に差分法を適用する場合,
2階微分が含まれることに注意する必要がある.

以下の波動方程式
\begin{align}
\frac{\partial^2 p}{\partial t^2} =c^2 \nabla^2 p
\end{align}
について,時間ステップ幅$\Delta t$,空間格子間隔$\Delta h$とおいて,
差分法を適用する.
\begin{align}
\begin{aligned}
&\frac{1}{c^2} \frac{p(x, y, z; t)-2p(x, y, z; t-\Delta t)+p(x, y, z; t-2\Delta t)}{(\Delta t)^2} \\
&=c^2\left\{\frac{p(x+\Delta x, y, z; t-\Delta t)-2p(x, y, z; t-\Delta t)+p(x-\Delta x, y, z; t-\Delta t)}{(\Delta h)^2} \right.\\
&+\frac{p(x, y+\Delta y, z; t-\Delta t)-2p(x, y, z; t-\Delta t)+p(x, y-\Delta y, z; t-\Delta t)}{(\Delta h)^2} \\
&\left.+\frac{p(x, y, z+\Delta z; t-\Delta t)-2p(x, y, z; t-\Delta t)+p(x, y, z-\Delta z; t-\Delta t)}{(\Delta h)^2}\right\}
\end{aligned}
\end{align}
整理すると,
\begin{align}
p(x, y, z; t)=\frac{(c\Delta t)^2}{(\delta h)^2}\{p(x+\Delta x, y, z; t-\Delta t)+\cdots\}+2p(x, y, z; t-\Delta t)-p(x, y, z; t-2\Delta t)
\end{align}
となり,時間ステップの更新式が得られた.

\begin{comment}

\section{テンプレート Template}

これはテストです。

This is a typeset template on \LaTeX.

\begin{quotation}
This is a quotation. 

これは引用文です。
\end{quotation}

Use \verb|\[\]| to show an expression:
\[a^2+b^2=c^2\]

The alternative is \verb|align*|:
\begin{align*}
a^2+b^2=c^2
\end{align*}

If expression numbers are needed, then use \verb|align|:
\begin{align}
a^2+b^2=c^2
\end{align}

\section{フォント Font}
Available font styles are:
\begin{itemize}
\item \textbf{Bold 太字}
\item \textit{Italic イタリック体}
\item \textrm{Roman ロマーン体}
\item \textsf{Sanserif サンセリフ体}
\item \textsc{Small Capital スモールキャピタル体}
\item \texttt{Typewriter タイプライタ体}
\item \textsl{Slant スラント体}
\item \emph{Emphasized 強調}
\end{itemize}

\section{ソースコード Source Code}
Source code is below:
\begin{lstlisting}[language=c]
int main()
{
    return 0;
}
\end{lstlisting}

\include{markdown/report}

\nocite{*} % 全ての参考文献を表示

%\bibliography{report}

\end{comment}

\begin{comment}

%%% gnuplot %%%
\begin{figure}[htbp]
    \centering
    \includegraphics{plot/graph.pdf}
    \caption{グラフタイトル}
    \label{fig:graph}
\end{figure}

%%% Inkscape %%%
\begin{figure}[htbp]
    \centering
    \def\svgwidth{200pt}
    \import{svg/}{draw.pdf_tex}
    \caption{図タイトル}
    \label{fig:draw}
\end{figure}

\end{comment}

\end{document}



\nocite{*} % 全ての参考文献を表示

%\bibliography{report}

\end{comment}

\begin{comment}

%%% gnuplot %%%
\begin{figure}[htbp]
    \centering
    \includegraphics{plot/graph.pdf}
    \caption{グラフタイトル}
    \label{fig:graph}
\end{figure}

%%% Inkscape %%%
\begin{figure}[htbp]
    \centering
    \def\svgwidth{200pt}
    \import{svg/}{draw.pdf_tex}
    \caption{図タイトル}
    \label{fig:draw}
\end{figure}

\end{comment}

\end{document}



\nocite{*} % 全ての参考文献を表示

%\bibliography{report}

\end{comment}

\begin{comment}

%%% gnuplot %%%
\begin{figure}[htbp]
    \centering
    \includegraphics{plot/graph.pdf}
    \caption{グラフタイトル}
    \label{fig:graph}
\end{figure}

%%% Inkscape %%%
\begin{figure}[htbp]
    \centering
    \def\svgwidth{200pt}
    \import{svg/}{draw.pdf_tex}
    \caption{図タイトル}
    \label{fig:draw}
\end{figure}

\end{comment}

\end{document}



\nocite{*} % 全ての参考文献を表示

%\bibliography{report}

\end{comment}

\begin{comment}

%%% gnuplot %%%
\begin{figure}[htbp]
    \centering
    \includegraphics{plot/graph.pdf}
    \caption{グラフタイトル}
    \label{fig:graph}
\end{figure}

%%% Inkscape %%%
\begin{figure}[htbp]
    \centering
    \def\svgwidth{200pt}
    \import{svg/}{draw.pdf_tex}
    \caption{図タイトル}
    \label{fig:draw}
\end{figure}

\end{comment}

\end{document}

